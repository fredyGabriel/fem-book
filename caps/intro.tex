%!TEX root = ../principal.tex

\setchapterimage[7.5cm]{figs/intro/highway}
\setchapterpreamble[u]{\margintoc}


\chapter{Introducción} \label{chap:intro}

\begin{kaobox}
	``Los científicos investigan lo que ya es; los ingenieros crean lo que nunca ha sido”
\begin{flushright}
	Albert Einstein
\end{flushright}
\end{kaobox}


\section{Lista de ideas}

\begin{enumerate}
	\item El libro debe tener su página web
	\item Me gustaría que gran parte sea open source, por ejemplo los códigos de python, podría ser también los Tikz de las figuras y otros deberían estar disponible para todos en GitHub. Pero el formato impreso (o digital completo) y los videos me gustaría monetizar;
	\item Cada ejercicio o alguna parte importante, puede tener un video asociado, y puesto en el libro como QR el enlace del video.
	\item Como imagen del capítulo se podría utilizar fotos de estructuras importantes del Paraguay.
	\item Cada capítulo pordría empezar con una frase inspiradora de ingenieros famosos, seguida de un resumen del capítulo.
	\item Usar cuadros de recordatorios a los márgenes para recordar algún concepto o parar repetir brevemento lo ya dicho en una sección anterior.
	\item El punto central es facilitar al máximo la lectura y el acceso de los recursos del libro.
\end{enumerate}


\section{En este capítulo}

\begin{enumerate}
	\item Introducción a los elementos finitos
	\item Notas históricas
	\item Aplicaciones
\end{enumerate}

\section{En este o en otro capítulo}

\begin{itemize}
	\item Base matemática
	\item Ecuaciones diferenciales y condiciones de frontera
	\item Álgebra lineal
	\item Cálculo variacional
\end{itemize}