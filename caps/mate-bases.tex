%!TEX root = ../main.tex
\setchapterimage[7.5cm]{figs/mate-bases/rotonda_czu1}
\setchapterpreamble[u]{\margintoc}

\chapter{Bases matemáticas} \label{chap:mate}

\begin{kaobox}
	``La naturaleza está escrita en lenguaje matemático"
	\begin{flushright}
		Galileo Galilei
	\end{flushright}
\end{kaobox}

Breve resumen histórico, basado en \cite{GoodVibrations01}

Alrededor del año 1670 el Cálculo fue inventado por Newton y/o Leibniz quienes tuvieron
la conocida controversia por los créditos de la creación de tan importante rama de la
matemática \cite{guicciardini2003reading}.

En 1687 fue publicado Philosophiæ naturalis principia mathematica (en latín) de Sir Isaac Newton \cite{newton1987}.

1696 Johann Bernoulli, problema de la braquistócrona.

1733 Euler abordó el problema de la braquistócrona.

1743 Principio de D'Alembert

1755 Lagrange a los 19 años, problema de la tautócrona.

1756 Euler, cálculo de variaciones.

1788 Lagrange, Mecánica Analítica

1834 Principio de Hamilton

\section{Cálculo variacional}

\subsection{Máximos, mínimos y puntos silla}

Es oportuno repasar algunos conceptos del cálculo diferencial. Las nociones de máximos,
mínimos y puntos silla son fundamentales en ingeniería.

En el análisis matemático, la identificación de máximos, mínimos y puntos silla de una función es crucial para comprender su comportamiento y sus características importantes. Estos puntos, conocidos colectivamente como puntos críticos, nos proporcionan información valiosa sobre dónde la función alcanza sus valores más altos y más bajos, o dónde exhibe un comportamiento de "silla de montar".

\subsubsection{Conceptos}

Consideremos una función $f(x,y)$ de dos variables.

\begin{itemize}
	\item \textbf{Máximo local}: Un punto $(x_0, y_0)$ es un máximo local de $f$ si $f(x_0, y_0) \ge f(x, y)$ para todos los puntos $(x, y)$ en una pequeña vecindad alrededor de  $(x_0, y_0)$.  En otras palabras, la función alcanza un valor máximo en $(x_0, y_0)$ dentro de esa vecindad.
	
	\item \textbf{Mínimo local}: Un punto $(x_0, y_0)$ es un mínimo local de $f$ si $f(x_0, y_0) \le f(x, y)$ para todos los puntos $(x, y)$ en una pequeña vecindad alrededor de  $(x_0, y_0)$. Es decir, la función alcanza un valor mínimo en $(x_0, y_0)$ dentro de esa vecindad.
	
	\item \textbf{Punto silla}: Un punto $(x_0, y_0)$ es un punto silla de $f$ si no es ni un máximo local ni un mínimo local. En un punto silla, la función tiene un comportamiento similar al de una silla de montar, donde aumenta en algunas direcciones y disminuye en otras.
\end{itemize}

\subsubsection{Expresiones Matemáticas}

Para encontrar los puntos críticos de una función $f(x,y)$, se deben seguir los siguientes pasos:

\begin{enumerate}
	\item \textbf{Calcular las derivadas parciales}:  Calcula las derivadas parciales de primer orden de $f$ con respecto a $x$ e $y$:
	
	$$\frac{\partial f}{\partial x} \quad \text{y} \quad \frac{\partial f}{\partial y}$$
	
	\item \textbf{Encontrar los puntos críticos}:  Encuentra los puntos $(x_0, y_0)$ donde ambas derivadas parciales son cero:
	
	$$\frac{\partial f}{\partial x}(x_0, y_0) = 0 \quad \text{y} \quad \frac{\partial f}{\partial y}(x_0, y_0) = 0$$
	
	\item \textbf{Calcular la matriz Hessiana}: Calcula la matriz Hessiana de $f$ en cada punto crítico $(x_0, y_0)$:
	
	$$H(x_0, y_0) = \begin{bmatrix} 
		\dfrac{\partial^2 f}{\partial x^2}(x_0, y_0) & \dfrac{\partial^2 f}{\partial x \partial y}(x_0, y_0) \\[3mm]
		\dfrac{\partial^2 f}{\partial y \partial x}(x_0, y_0) & \dfrac{\partial^2 f}{\partial y^2}(x_0, y_0) 
	\end{bmatrix}$$
	
	\item \textbf{Determinar la naturaleza de los puntos críticos}:
	\begin{itemize}
		\item Si el determinante de la matriz Hessiana es positivo ($|H| > 0$) y la segunda derivada parcial con respecto a $x$ es positiva ($\dfrac{\partial^2 f}{\partial x^2} > 0$), entonces $(x_0, y_0)$ es un \textit{mínimo local}.
		
		\item Si el determinante de la matriz Hessiana es positivo ($|H| > 0$) y la segunda derivada parcial con respecto a $x$ es negativa ($\dfrac{\partial^2 f}{\partial x^2} < 0$), entonces $(x_0, y_0)$ es un \textit{máximo local}.
		
		\item Si el determinante de la matriz Hessiana es negativo ($|H| < 0$), entonces $(x_0, y_0)$ es un \textit{punto silla}.
		
		\item Si el determinante de la matriz Hessiana es cero ($|H| = 0$), la prueba es inconclusa y se necesitan métodos adicionales para determinar la naturaleza del punto crítico.
	\end{itemize}
\end{enumerate}

\subsubsection{Ejemplos de Aplicación}

\begin{example}
	
	Encontrar los puntos críticos de la función $f(x,y) = x^2 + y^2 - 2x - 4y + 5$.
	
	\textbf{Solución}:
	
	\begin{enumerate}
		\item Derivadas parciales:
		
		$$\dfrac{\partial f}{\partial x} = 2x - 2, \quad \dfrac{\partial f}{\partial y} = 2y - 4$$
		
		\item Puntos críticos:
		
		Resolviendo el sistema de ecuaciones $\frac{\partial f}{\partial x} = 0$ y $\frac{\partial f}{\partial y} = 0$, obtenemos el punto crítico $(1, 2)$.
		
		\item Matriz Hessiana:
		
		$$H(1, 2) = \begin{bmatrix} 2 & 0 \\ 0 & 2 \end{bmatrix}$$
		
		\item Naturaleza del punto crítico:
		
		El determinante de la matriz Hessiana es $|H| = 4 > 0$ y $\dfrac{\partial^2 f}{\partial x^2} = 2 > 0$. Por lo tanto, el punto $(1, 2)$ es un \textit{mínimo local}.
	\end{enumerate}
\end{example}


\begin{example}
	
	Encontrar los puntos críticos de la función $f(x,y) = x^3 - 3xy^2$.
	
	\textbf{Solución}:
	
	\begin{enumerate}
		\item Derivadas parciales:
		$$\frac{\partial f}{\partial x} = 3x^2 - 3y^2, \quad \frac{\partial f}{\partial y} = -6xy$$
		
		\item Puntos críticos:
		
		Resolviendo el sistema de ecuaciones $\frac{\partial f}{\partial x} = 0$ y $\frac{\partial f}{\partial y} = 0$, obtenemos el punto crítico $(0, 0)$.
		
		\item Matriz Hessiana:
		$$H(0, 0) = \begin{bmatrix} 0 & 0 \\ 0 & 0 \end{bmatrix}$$
		
		\item Naturaleza del punto crítico:
		
		El determinante de la matriz Hessiana es $|H| = 0$. La prueba es inconclusa. En este caso, se puede observar que la función tiene un comportamiento de silla de montar en $(0,0)$.
	\end{enumerate}
\end{example}

Estos ejemplos ilustran cómo identificar y clasificar los puntos críticos de una función de dos variables.  Los máximos, mínimos y puntos silla son conceptos fundamentales en el análisis matemático y tienen aplicaciones en diversas áreas de la ciencia y la ingeniería.


\subsection{Ecuación de Euler-Lagrange}

El cálculo variacional es una rama del análisis matemático que se ocupa de encontrar funciones que maximizan o minimizan ciertas cantidades, llamadas funcionales. Un funcional es una función que toma como argumento, también, una función y devuelve un número real. Un ejemplo clásico de un funcional es la longitud de una curva entre dos puntos.

La ecuación de Euler-Lagrange es una herramienta fundamental en el cálculo variacional.  Proporciona una condición necesaria que debe cumplir una función para ser un extremo (máximo o mínimo) de un funcional.

\subsubsection{La Ecuación de Euler-Lagrange}

Consideremos un funcional de la forma:

\begin{equation}
	J[y] = \int_a^b L(x, y(x), y'(x)) \, dx
\end{equation}

donde:

\begin{itemize}
	\item $y(x)$ es la función que queremos encontrar.
	\item $y'(x)$ es la derivada de $y(x)$.
	\item $L(x, y, y')$ es una función dada, llamada Lagrangiana.
\end{itemize}

La ecuación de Euler-Lagrange establece que si $y(x)$ es un extremo del funcional $J[y]$, entonces satisface la siguiente ecuación diferencial:

\begin{equation}
	\frac{\partial L}{\partial y} - \frac{d}{dx} \left( \frac{\partial L}{\partial y'} \right) = 0
\end{equation}

\subsubsection{Ejemplo de Aplicación: La Braquistócrona}

Un ejemplo clásico de la aplicación de la ecuación de Euler-Lagrange es el problema de la braquistócrona.  Este problema consiste en encontrar la curva que une dos puntos A y B en un plano vertical, de tal manera que una partícula que se desliza sin fricción bajo la influencia de la gravedad recorra la curva en el menor tiempo posible.

%\begin{figure}[h]
%	\centering
%	\includegraphics[width=0.6\textwidth]{braquistocrona.png} % Reemplaza con una imagen de la braquistócrona
%	\caption{Curva braquistócrona entre dos puntos A y B.}
%\end{figure}

En este caso, el funcional que queremos minimizar es el tiempo de recorrido, que se puede expresar como:

\begin{equation}
	T[y] = \int_a^b \frac{\sqrt{1 + (y'(x))^2}}{\sqrt{2gy(x)}} \, dx
\end{equation}

donde $g$ es la aceleración debida a la gravedad.

La Lagrangiana correspondiente es:

\begin{equation}
	L(x, y, y') = \frac{\sqrt{1 + (y'(x))^2}}{\sqrt{2gy(x)}}
\end{equation}

Aplicando la ecuación de Euler-Lagrange, se puede demostrar que la curva que minimiza el tiempo de recorrido es una cicloide.

\section{Conclusión}

La ecuación de Euler-Lagrange es una herramienta fundamental en el cálculo variacional y tiene aplicaciones en diversas áreas de la física, la ingeniería y la economía.  Permite encontrar funciones que optimizan funcionales, lo que tiene implicaciones en la resolución de problemas de optimización y en la modelización de fenómenos físicos.

\section{Álgebra lineal}
\subsection{Operaciones con matrices}
\subsection{Espacios vectoriales}
\subsection{Transformación lineal}
\subsection{Cambio de base}
\subsection{Espacios de Hilbert}
\subsection{Valores y vectores propios}