%!TEX root = ../principal.tex
\setchapterimage[7.5cm]{figs/mate-bases/rotonda_czu1}
\setchapterpreamble[u]{\margintoc}

\chapter{Bases matemáticas} \label{chap:mate}

\begin{kaobox}
	``La naturaleza está escrita en lenguaje matemático"
	\begin{flushright}
		Galileo Galilei
	\end{flushright}
\end{kaobox}

Breve resumen histórico, basado en \cite{GoodVibrations01}

Alrededor del año 1670 el Cálculo fue inventado por Newton y/o Leibniz quienes tuvieron
la conocida controversia por los créditos de la creación de tan importante rama de la
matemática \cite{guicciardini2003reading}.

En 1687 fue publicado Philosophiæ naturalis principia mathematica (en latín) de Sir Isaac Newton \cite{newton1987}.

1696 Johann Bernoulli, problema de la braquistócrona.

1733 Euler abordó el problema de la braquistócrona.

1743 Principio de D'Alembert

1755 Lagrange a los 19 años, problema de la tautócrona.

1756 Euler, cálculo de variaciones.

1788 Lagrange, Mecánica Analítica

1834 Principio de Hamilton

\section{Cálculo variacional}

\subsection{Máximos, mínimos y puntos silla}

Es oportuno repasar algunos conceptos del cálculo diferencial. Las nociones de máximos,
mínimos y puntos silla son fundamentales en ingeniería.

$f(c)$ es un \textit{máximo relativo} de una función $f(x)$ si $f(c) \ge f(x)$ para 
todos los valores $x$ en la \textit{vecindad} de $c$.


\subsection{Ecuación de Euler-Lagrange}

\section{Álgebra lineal}
\subsection{Operaciones con matrices}
\subsection{Espacios vectoriales}
\subsection{Transformación lineal}
\subsection{Cambio de base}
\subsection{Espacios de Hilbert}
\subsection{Valores y vectores propios}